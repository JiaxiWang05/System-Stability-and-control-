\documentclass[12pt]{article}
\usepackage{amsmath, amssymb, hyperref}

% Hyperlink setup
\hypersetup{
    colorlinks=true,
    linkcolor=blue,
    urlcolor=blue,
    pdftitle={Questions Document},
    bookmarksnumbered=true,
    pdfpagemode=UseOutlines,
    pdfstartview=FitH,
    pdfauthor={Your Name},
    pdfsubject={Exercise Solutions}
}

\begin{document}

% Title
\title{Exercise Solutions}
\author{Your Name}
\date{\today}
\maketitle

% Table of Contents
\tableofcontents
\newpage

% Question 1
\section*{Question 1: Solving a Differential Equation Using Laplace Transform}
\addcontentsline{toc}{section}{Question 1: Solving a Differential Equation Using Laplace Transform}

\textbf{Exercise Overview:}

\textbf{Given Differential Equation:}  
\[
2\frac{d^2y(t)}{dt^2} + 4\frac{dy(t)}{dt} + 2y(t) = f(t)
\]

\textbf{Tasks:}
\begin{enumerate}
    \item Find the Laplace Transform of the equation.
    \item Derive the transfer function of the system.
    \item Determine the zeros and poles of the transfer function.
\end{enumerate}

---

\subsection*{Step 1: Laplace Transform of the Differential Equation}
The Laplace Transform converts the time-domain differential equation into an algebraic equation in the Laplace domain. The key properties used here include:
\[
\mathcal{L}\left[\frac{d^n y(t)}{dt^n}\right] = s^n Y(s),
\]
where \( Y(s) \) is the Laplace Transform of \( y(t) \), assuming zero initial conditions.

\textbf{Given Differential Equation:}
\[
2\frac{d^2y(t)}{dt^2} + 4\frac{dy(t)}{dt} + 2y(t) = f(t)
\]

Applying the Laplace Transform:
\begin{align*}
    \mathcal{L}\left[2\frac{d^2y(t)}{dt^2}\right] &= 2s^2 Y(s), \\
    \mathcal{L}\left[4\frac{dy(t)}{dt}\right] &= 4s Y(s), \\
    \mathcal{L}[2y(t)] &= 2Y(s), \\
    \mathcal{L}[f(t)] &= F(s).
\end{align*}

Substituting these into the equation:
\[
2s^2 Y(s) + 4s Y(s) + 2Y(s) = F(s).
\]

Combine terms involving \( Y(s) \):
\[
Y(s) \big( 2s^2 + 4s + 2 \big) = F(s).
\]

This is the Laplace-transformed equation.

---

\subsection*{Step 2: Transfer Function}
The transfer function \( H(s) \) is defined as:
\[
H(s) = \frac{Y(s)}{F(s)}.
\]

From the Laplace-transformed equation:
\[
Y(s) = \frac{F(s)}{2s^2 + 4s + 2}.
\]

The transfer function becomes:
\[
H(s) = \frac{1}{2s^2 + 4s + 2}.
\]

Simplify the denominator:
\[
H(s) = \frac{1}{2(s^2 + 2s + 1)} = \frac{1}{2(s+1)^2}.
\]

---

\subsection*{Step 3: Zeros and Poles}
\begin{enumerate}
    \item \textbf{Zeros:}  
    Zeros are the roots of the numerator of the transfer function. Since the numerator is \( 1 \), there are no zeros.

    \item \textbf{Poles:}  
    Poles are the roots of the denominator \( 2(s+1)^2 \):
    \[
    2(s+1)^2 = 0 \implies s+1 = 0 \implies s = -1.
    \]
    Thus, there is a single pole at \( s = -1 \), but it is a repeated pole due to the squared term.
\end{enumerate}

---

\subsection*{Final Results}
\begin{enumerate}
    \item \textbf{Laplace Transform} of the differential equation:
    \[
    Y(s) \big( 2s^2 + 4s + 2 \big) = F(s).
    \]

    \item \textbf{Transfer Function:}
    \[
    H(s) = \frac{1}{2(s+1)^2}.
    \]

    \item \textbf{Zeros and Poles:}
    \begin{itemize}
        \item \textbf{Zeros:} None.
        \item \textbf{Poles:} \( s = -1 \) (repeated pole).
    \end{itemize}
\end{enumerate}

\newpage

% Question 2 (Placeholder)\section*{Question 2: Finding the Transfer Function of a Second-Order System}
\addcontentsline{toc}{section}{Question 2: Finding the Transfer Function of a Second-Order System}

\textbf{Problem Overview:}

\textbf{Given Second-Order System:}
\[
\alpha \frac{d^2y(t)}{dt^2} + \beta \frac{dy(t)}{dt} + \gamma y(t) = 2r(t)
\]

\textbf{Objective:}
Find the \textbf{transfer function} of the system.

---

\subsection*{Step 1: Recall the Laplace Transform}

The Laplace Transform is applied to each term in the time-domain equation to convert it into an algebraic equation in the \( s \)-domain. We use the following rules:
\begin{align*}
    \mathcal{L}\left[\frac{d^2 y(t)}{dt^2}\right] &= s^2 Y(s) \quad \text{(for zero initial conditions)}, \\
    \mathcal{L}\left[\frac{dy(t)}{dt}\right] &= s Y(s), \\
    \mathcal{L}[y(t)] &= Y(s), \\
    \mathcal{L}[r(t)] &= R(s).
\end{align*}

---

\subsection*{Step 2: Apply the Laplace Transform to the Given Equation}

The given system is:
\[
\alpha \frac{d^2y(t)}{dt^2} + \beta \frac{dy(t)}{dt} + \gamma y(t) = 2r(t)
\]

Applying the Laplace Transform to each term:

\begin{enumerate}
    \item For \( \alpha \frac{d^2y(t)}{dt^2} \):  
    \[
    \mathcal{L}\left[\alpha \frac{d^2y(t)}{dt^2}\right] = \alpha s^2 Y(s)
    \]

    \item For \( \beta \frac{dy(t)}{dt} \):  
    \[
    \mathcal{L}\left[\beta \frac{dy(t)}{dt}\right] = \beta s Y(s)
    \]

    \item For \( \gamma y(t) \):  
    \[
    \mathcal{L}[\gamma y(t)] = \gamma Y(s)
    \]

    \item For \( 2r(t) \):  
    \[
    \mathcal{L}[2r(t)] = 2R(s)
    \]
\end{enumerate}

Substituting these into the given equation:
\[
\alpha s^2 Y(s) + \beta s Y(s) + \gamma Y(s) = 2R(s)
\]

---

\subsection*{Step 3: Simplify the Equation}

Factor \( Y(s) \) on the left-hand side:
\[
Y(s) \big( \alpha s^2 + \beta s + \gamma \big) = 2R(s)
\]

Rearranging for \( \frac{Y(s)}{R(s)} \):
\[
\frac{Y(s)}{R(s)} = \frac{2}{\alpha s^2 + \beta s + \gamma}
\]

---

\subsection*{Step 4: Write the Transfer Function}

The transfer function \( H(s) \) is defined as:
\[
H(s) = \frac{Y(s)}{R(s)}
\]

From the simplified equation:
\[
H(s) = \frac{2}{\alpha s^2 + \beta s + \gamma}
\]

---

\subsection*{Final Result}

The transfer function for the system is:
\[
H(s) = \frac{2}{\alpha s^2 + \beta s + \gamma}
\]

---

\subsection*{Explanation of the Process}

\begin{enumerate}
    \item The Laplace Transform simplifies differential equations by converting time-domain derivatives into algebraic terms involving \( s \).
    \item By substituting the Laplace transforms of each term, the equation becomes a polynomial in \( s \) with \( Y(s) \) (the Laplace Transform of \( y(t) \)) and \( R(s) \) (the Laplace Transform of \( r(t) \)) as variables.
    \item Factoring out \( Y(s) \), we isolate the ratio \( \frac{Y(s)}{R(s)} \), which defines the transfer function.
    \item The transfer function describes the relationship between the input \( r(t) \) and output \( y(t) \) in the \( s \)-domain.
\end{enumerate}


\newpage

\section*{Question 4: Stability Analysis and Transfer Function of a Second-Order System}
\addcontentsline{toc}{section}{Question 4: Stability Analysis and Transfer Function of a Second-Order System}

Let’s go through the exercise and solution step by step in detail.

---

\subsection*{Problem Overview}

\textbf{System Given:}
\[
4 \frac{d^2y(t)}{dt^2} - 3K \frac{dy(t)}{dt} + y(t) = f(t)
\]

\textbf{Tasks:}
\begin{enumerate}
    \item Find the \textbf{Laplace transform} of the given equation.
    \item Determine the \textbf{transfer function} of the system.
    \item Check if the system is \textbf{stable} for \( K = \frac{4}{3} \).
\end{enumerate}

---

\subsection*{Step 1: Laplace Transform of the Equation}

The Laplace Transform converts the differential equation from the time domain to the Laplace domain. For zero initial conditions, we use:
\begin{align*}
    \mathcal{L}\left[\frac{d^2y(t)}{dt^2}\right] &= s^2 Y(s), \\
    \mathcal{L}\left[\frac{dy(t)}{dt}\right] &= s Y(s), \\
    \mathcal{L}[y(t)] &= Y(s), \\
    \mathcal{L}[f(t)] &= F(s).
\end{align*}

\textbf{Given equation:}
\[
4 \frac{d^2y(t)}{dt^2} - 3K \frac{dy(t)}{dt} + y(t) = f(t)
\]

Applying the Laplace Transform:
\[
4 s^2 Y(s) - 3K s Y(s) + Y(s) = F(s)
\]

---

\subsection*{Step 2: Simplify the Equation}

Factor \( Y(s) \) on the left-hand side:
\[
Y(s) \big( 4 s^2 - 3K s + 1 \big) = F(s)
\]

Rearrange to solve for \( Y(s) \):
\[
Y(s) = \frac{F(s)}{4 s^2 - 3K s + 1}
\]

---

\subsection*{Step 3: Transfer Function}

The transfer function \( H(s) \) is defined as:
\[
H(s) = \frac{Y(s)}{F(s)}
\]

From the equation:
\[
H(s) = \frac{1}{4 s^2 - 3K s + 1}
\]

---

\subsection*{Step 4: System Stability}

The system's stability depends on the location of the \textbf{poles}, which are the roots of the denominator \( 4 s^2 - 3K s + 1 \). The roots are found using the quadratic formula:
\[
s = \frac{-b \pm \sqrt{b^2 - 4ac}}{2a}
\]

Here:
\begin{align*}
    a &= 4, \\
    b &= -3K, \\
    c &= 1.
\end{align*}

Substituting into the formula:
\[
s = \frac{-(-3K) \pm \sqrt{(-3K)^2 - 4(4)(1)}}{2(4)}
\]
\[
s = \frac{3K \pm \sqrt{9K^2 - 16}}{8}
\]

---

\subsection*{Case: \( K = \frac{4}{3} \)}

Substitute \( K = \frac{4}{3} \) into the equation:
\[
s = \frac{3 \cdot \frac{4}{3} \pm \sqrt{9 \left(\frac{4}{3}\right)^2 - 16}}{8}
\]

Simplify \( 3K = 4 \):
\[
s = \frac{4 \pm \sqrt{16 - 16}}{8}
\]

Simplify further:
\[
s = \frac{4}{8} = \frac{1}{2}
\]

---

\subsection*{Step 5: Conclusion on Stability}

Since one of the poles (\( s = \frac{1}{2} \)) has a positive real part, the system is \textbf{unstable} for \( K = \frac{4}{3} \).

---

\subsection*{Final Results}

\begin{enumerate}
    \item \textbf{Laplace Transform:}
    \[
    4 s^2 Y(s) - 3K s Y(s) + Y(s) = F(s)
    \]

    \item \textbf{Transfer Function:}
    \[
    H(s) = \frac{1}{4 s^2 - 3K s + 1}
    \]

    \item \textbf{Stability:}
    For \( K = \frac{4}{3} \), the system is \textbf{unstable} because one of the poles (\( s = \frac{1}{2} \)) has a positive real part.
\end{enumerate}
\section*{Question 5: Derivation of the Second-Order Differential Equation for a Series RLC Circuit}
\addcontentsline{toc}{section}{Question 5: Derivation of the Second-Order Differential Equation for a Series RLC Circuit}

Let’s go step by step and in great detail explain the derivation of the second-order differential equation for the \textbf{series RLC circuit}.

---

\subsection*{Circuit Description}

We are analyzing a \textbf{series RLC circuit}, which consists of:
\begin{itemize}
    \item \textbf{Resistor (R):} Voltage drop across the resistor is proportional to the current, \( v_R = R i(t) \).
    \item \textbf{Inductor (L):} Voltage drop across the inductor is proportional to the rate of change of current, \( v_L = L \frac{di(t)}{dt} \).
    \item \textbf{Capacitor (C):} Voltage across the capacitor is proportional to the charge, \( v_C = v(t) \), where \( q(t) = C v(t) \).
\end{itemize}

Key points:
\begin{itemize}
    \item The current \( i(t) \) is the same through all components because they are in series.
    \item The total voltage in the loop is distributed across the three components.
\end{itemize}

---

\subsection*{Step 1: Kirchhoff's Voltage Law (KVL)}

Using Kirchhoff’s Voltage Law, the sum of voltages across all components in a closed loop equals zero:
\[
v_R + v_L + v_C = 0
\]

Substitute the expressions for the voltage across each component:
\[
R i(t) + L \frac{di(t)}{dt} + v(t) = 0 \tag{1}
\]

Here:
\begin{itemize}
    \item \( R i(t) \): Voltage drop across the resistor.
    \item \( L \frac{di(t)}{dt} \): Voltage drop across the inductor.
    \item \( v(t) \): Voltage across the capacitor.
\end{itemize}

---

\subsection*{Step 2: Relating Current \( i(t) \) to Voltage \( v(t) \)}

The relationship between the current \( i(t) \) and the voltage across the capacitor \( v(t) \) comes from the definition of capacitance:
\[
i(t) = C \frac{dv(t)}{dt} \tag{2}
\]

This equation states that the current flowing through the capacitor is proportional to the rate of change of the voltage across it.

---

\subsection*{Step 3: Substitute \( i(t) = C \frac{dv(t)}{dt} \) into KVL}

Substitute \( i(t) = C \frac{dv(t)}{dt} \) into the KVL equation:
\[
R \left(C \frac{dv(t)}{dt}\right) + L \frac{d}{dt} \left(C \frac{dv(t)}{dt}\right) + v(t) = 0
\]

Expand the terms:
\begin{itemize}
    \item Resistor term:
    \[
    R C \frac{dv(t)}{dt}
    \]

    \item Inductor term:
    \[
    L \frac{d}{dt} \left(C \frac{dv(t)}{dt}\right) = L C \frac{d^2v(t)}{dt^2}
    \]

    \item Capacitor term:
    \[
    v(t)
    \]
\end{itemize}

Substituting back:
\[
L C \frac{d^2v(t)}{dt^2} + R C \frac{dv(t)}{dt} + v(t) = 0 \tag{3}
\]

---

\subsection*{Step 4: Normalize the Equation}

To simplify, divide through by \( C \):
\[
L \frac{d^2v(t)}{dt^2} + R \frac{dv(t)}{dt} + \frac{1}{C} v(t) = 0
\]

Finally, divide through by \( L \) to express the equation in a normalized form:
\[
\frac{d^2v(t)}{dt^2} + \frac{R}{L} \frac{dv(t)}{dt} + \frac{1}{L C} v(t) = 0 \tag{4}
\]

---

\subsection*{Final Second-Order Differential Equation}

The final equation is:
\[
\frac{d^2v(t)}{dt^2} + \frac{R}{L} \frac{dv(t)}{dt} + \frac{1}{L C} v(t) = 0 \tag{5}
\]

---

\subsection*{Explanation of Terms}

\begin{enumerate}
    \item \textbf{\(\frac{d^2v(t)}{dt^2}\):}
    Represents the acceleration (second derivative) of the voltage across the capacitor, arising from the inductance (\( L \)) and inertia for current changes.

    \item \textbf{\(\frac{R}{L} \frac{dv(t)}{dt}\):}
    Represents the damping effect in the circuit caused by resistance (\( R \)). Larger \( R \) increases damping, reducing oscillations.

    \item \textbf{\(\frac{1}{L C} v(t)\):}
    Represents the restoring force due to capacitance (\( C \)) and inductance (\( L \)), determining the natural oscillatory behavior of the system.
\end{enumerate}

---

\subsection*{Physical Meaning of the Equation}

The second-order differential equation describes the behavior of the voltage across the capacitor in the RLC circuit:
\begin{itemize}
    \item The circuit oscillates due to the interplay of inductance (\( L \)) and capacitance (\( C \)).
    \item The resistance (\( R \)) damps the oscillations, leading to energy loss over time.
    \item The natural frequency of oscillation is given by:
    \[
    \omega_n = \sqrt{\frac{1}{L C}} \tag{6}
    \]
    \item The damping ratio is given by:
    \[
    \zeta = \frac{R}{2} \sqrt{\frac{C}{L}} \tag{7}
    \]
\end{itemize}

---

\subsection*{Summary}

The derived equation:
\[
\frac{d^2v(t)}{dt^2} + \frac{R}{L} \frac{dv(t)}{dt} + \frac{1}{L C} v(t) = 0
\]
is a second-order differential equation describing the voltage across the capacitor in a series RLC circuit. It shows how the system oscillates, is damped by resistance, and is governed by the natural frequency \( \omega_n \).

---

\newpage
\section*{Question 6: Derivation of the Differential Equation for a Parallel RLC Circuit}
\addcontentsline{toc}{section}{Question 6: Derivation of the Differential Equation for a Parallel RLC Circuit}

Let’s go through the \textbf{Parallel RLC Circuit} analysis step by step in \textbf{great detail}.

---

\subsection*{Step 1: Parallel RLC Circuit Description}

In a parallel RLC circuit:
\begin{itemize}
    \item The \textbf{resistor (\(R\))}, \textbf{inductor (\(L\))}, and \textbf{capacitor (\(C\))} are connected in parallel.
    \item The voltage across all three components is the same and is denoted as \(v(t)\).
    \item The total current flowing into the circuit is the sum of the currents through each branch (capacitor, resistor, and inductor).
\end{itemize}

---

\subsection*{Step 2: Kirchhoff’s Current Law (KCL)}

Using Kirchhoff's Current Law (KCL), the total current entering the parallel circuit equals the sum of the currents through the resistor, capacitor, and inductor:
\[
i_C + i_R + i_L = 0
\]

\textbf{Current Expressions}:
\begin{enumerate}
    \item Current through the capacitor (\(i_C\)):
    \[
    i_C = C \frac{dv(t)}{dt}
    \]

    \item Current through the resistor (\(i_R\)):
    \[
    i_R = \frac{v(t)}{R}
    \]

    \item Current through the inductor (\(i_L\)):
    \[
    i_L = \frac{1}{L} \int_0^t v(t) \, dt
    \]
\end{enumerate}

Substituting these expressions into the KCL equation:
\[
C \frac{dv(t)}{dt} + \frac{v(t)}{R} + \frac{1}{L} \int_0^t v(t) \, dt = 0 \tag{1}
\]

---

\subsection*{Step 3: Eliminate the Integral Term}

To convert this into a differential equation, differentiate both sides with respect to time:
\[
\frac{d}{dt} \left[ C \frac{dv(t)}{dt} + \frac{v(t)}{R} + \frac{1}{L} \int_0^t v(t) \, dt \right] = 0
\]

Expanding each term:
\begin{itemize}
    \item The derivative of the capacitor current:
    \[
    \frac{d}{dt} \left( C \frac{dv(t)}{dt} \right) = C \frac{d^2v(t)}{dt^2}
    \]

    \item The derivative of the resistor current:
    \[
    \frac{d}{dt} \left( \frac{v(t)}{R} \right) = \frac{1}{R} \frac{dv(t)}{dt}
    \]

    \item The derivative of the inductor current:
    \[
    \frac{d}{dt} \left( \frac{1}{L} \int_0^t v(t) \, dt \right) = \frac{1}{L} v(t)
    \]
\end{itemize}

Substituting back into the equation:
\[
C \frac{d^2v(t)}{dt^2} + \frac{1}{R} \frac{dv(t)}{dt} + \frac{1}{L} v(t) = 0 \tag{2}
\]

---

\subsection*{Step 4: Final Differential Equation}

Rewriting the equation:
\[
\frac{d^2v(t)}{dt^2} + \frac{1}{RC} \frac{dv(t)}{dt} + \frac{1}{LC} v(t) = 0 \tag{3}
\]

This is the \textbf{second-order differential equation} describing the voltage dynamics in a \textbf{parallel RLC circuit}.

---

\subsection*{Step 5: Stability Analysis}

The stability and damping characteristics of the system are determined by the coefficients of this equation.

\textbf{General Form of a Second-Order Equation:}
\[
\ddot{v} + 2\zeta\omega_n \dot{v} + \omega_n^2 v = 0
\]

Comparing with:
\[
\ddot{v} + \frac{1}{RC} \dot{v} + \frac{1}{LC} v = 0
\]

\begin{itemize}
    \item The \textbf{natural frequency (\(\omega_n\))} is:
    \[
    \omega_n = \sqrt{\frac{1}{LC}}
    \]

    \item The \textbf{damping ratio (\(\zeta\))} is:
    \[
    \zeta = \frac{1}{2RC} \sqrt{LC}
    \]
\end{itemize}

---

\subsection*{Step 6: Overdamped, Critically Damped, and Underdamped Cases}

For a second-order system, the behavior depends on the discriminant of the characteristic equation:
\[
b^2 - 4ac
\]

Here, the coefficients are:
\begin{align*}
    a &= 1, \\
    b &= \frac{1}{RC}, \\
    c &= \frac{1}{LC}.
\end{align*}

The system is:
\begin{enumerate}
    \item \textbf{Overdamped:} \(b^2 > 4ac\)
    \[
    R^2C^2 > \frac{L}{4C}
    \]

    \item \textbf{Critically Damped:} \(b^2 = 4ac\)
    \[
    R = \sqrt{\frac{L}{4C}}
    \]

    \item \textbf{Underdamped:} \(b^2 < 4ac\)
    \[
    R^2C^2 < \frac{L}{4C}
    \]
\end{enumerate}

---

\subsection*{Step 7: Oscillation Frequency}

When the system is underdamped, it oscillates with a frequency less than the natural frequency (\(\omega_n\)). The damped oscillation frequency is:
\[
\omega_d = \sqrt{\frac{1}{LC} - \left(\frac{1}{2RC}\right)^2} \tag{4}
\]

---

\subsection*{Key Observations}

\begin{enumerate}
    \item \textbf{Stability:}
    The system is always stable if \(L > 0\), \(R > 0\), and \(C > 0\), because there are no poles in the right-half plane.

    \item \textbf{Damping:}
    The damping ratio depends on \(R\), \(L\), and \(C\). Smaller \(R\) leads to oscillatory behavior, while larger \(R\) leads to overdamping.

    \item \textbf{Oscillatory Behavior:}
    Oscillations occur only in the underdamped case, where \(R < \sqrt{\frac{L}{4C}}\).
\end{enumerate}

---



\end{document}
